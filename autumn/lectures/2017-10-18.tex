\section{Динамика}
\subsubsection*{Принцип детерминированности Ньютона}
\[ \v r_i(t) = \varphi_i(\v r_1, \ldots, \v r_N, \dot {\v r}_1, \ldots, \dot{\v r}_N, t_0, t)~~ \forall t_0 \]
\[ \ddot{\v r}_i(t) = \frac{d^2 \varphi_i}{dt^2} = f_i(\v r_1, \ldots, \v r_N, \dot {\v r}_1, \ldots, \dot{\v r}_N, t_0, t)  \]
\[ \v r_i(t_0) = f_i (\ldots, t)~~ \forall t_0 \]
\begin{equation}
\label{seven.one}
\ddot{\v r}_i(t_0) = f_i (\v r_1, \ldots, \v r_N, \dot{\v r}_1, \ldots, \dot{\v r}_N, t)~~ \forall t_0 
\end{equation}

\begin{xmp}
$ f = 0 \Rightarrow \ddot{\v r} = 0,~ \v r = \v r_0 + \dot{\v r}_0(t - t_0) $ \\
(Закон инерции Галилео-Ньютона); если $m_i$ - масса точки $\v r_i$
\[ m_i\ddot{\v r}_i = \v F_i;~~ \v F_i = m_i\v f_i \text{ --- \emph{сила}} \]
\end{xmp}
\subsubsection*{Преобразование Галилея}
$\v r \rightarrow r^* = \underbrace{A\v r}_{\text{Ортог. пр.}} + \v v_0 t + \v r_0,~~ t^* = t + t_0$ \\
$ A = const,~~ \v v_0 = const,~~ \v r_0 = const$

\subsubsection*{Принцип относительности Галилея}
\begin{flalign*}
& m_i \ddot{\v r}_i = \v F_i(\v r_1, \ldots \v r_N, \dot{\v r}_1, \ldots, \dot{\v r}_N, t) &\\
& m_i \ddot{\v r}_i^* = \v F_i(\v r_1^*, \ldots \v r_N^*, \dot{\v r}_1^*, \ldots, \dot{\v r}_N^*, t^*) &\\
& \frac{d\v r_i^*}{dt^*} = \frac{d \v r_i^*}{dt} \cdot 1 &\\
& \ddot{\v r}_i^* = A\ddot{\v r} \Rightarrow \v F_i^* = A \v F_i &\\
\end{flalign*}
Принцип относительности: 
\[ \v F_i^*(\v r_1^*, \ldots \v r_N^*, \dot{\v r}_1^*, \ldots, \dot{\v r}_N^*, t^*) = \v F_i(\v r_1^*, \ldots \v r_N^*, \dot{\v r}_1^*, \ldots, \dot{\v r}_N^*, t^*) \]
\begin{xmp}
$ n = 1 $:\\
$\v F = A \v F,~~ \forall A \Leftrightarrow \v F = 0$
\end{xmp}
\begin{xmp}
$r^* = \v r,~~ t^* = t - t_0,~~ t = t_0 \Rightarrow \v F_i(\ldots, t) = \v F_i(\ldots, 0)$
\end{xmp}

\subsubsection*{Закон равенства действия и противодействия}
\[ \v F_{ij} = -\v F_{ji},~~ \v F_{ij} \parallel \v r_j - \v r_i \]

\subsubsection*{Принцип суперпозиции}
\begin{flalign*}
& \v F_i = \sum\limits_{i \neq j} \v F_{ij} \text{ (Для замкнутых систем)} &\\
& \v F_i = \v F_i^{(e)} + \v F_i^{(i)} &\\
& \v F_i^{(e)} \text{ --- внешняя сила} &\\
& \v F_i^{(i)} \text{ --- внутренняя сила} &\\
& \text{Система неинерциальная} &\\
& \v w_i^{\text{абс}} = \v w_i^{\text{отн}} + \v w_i^{\text{пер}} + \v w_i^{\text{кор}} &\\
& m_i \ddot{\v \rho}_i = \v F_i + \v F_i^{\text{отн}} + \v F_i^{\text{пер}} &\\
& \v w_i^{\text{отн}} = \ddot{\v \rho}_i;~ \v F_i^{\text{отн}} = -m_i\v w_i^{\text{отн}};~ \v F_i^\text{пер} = -m_i(\v w_0 + [\v \varepsilon, \v \rho] + [\v \omega, [\v \omega, \v \rho_i]]) &\\
\end{flalign*}

\begin{df}
$ \v M_O = [\v r, \v F] $ --- момент силы $\v F$ относительно $O$
\end{df}
\begin{df}
$M_l = (\v M_O, \v l)$ --- момент силы $\v F$ относительно оси $\v l$
\end{df}

\begin{ass}
$M_l$ не зависит от выбора точки $O$.
\end{ass}
\begin{proof}
\begin{flalign*}
& M_l = (\v M_O, \v l) = ([\v r, \v F], \v l) = ([\v r' + \v{O'O}, \v F], \v l) = &\\
& = ([\v r', \v F], \v l) + (\underbrace{[\lambda\v l, \v F]}_0, \v l) \Rightarrow M_l = (\v M_O, \v l) &\\
\end{flalign*}
\end{proof}

\begin{df} 
$(\v F, d\v r)$ --- элементарная работа ($dA$, $d'A$, $\delta A$, $A_{\text{эл}}$)
\end{df}

\subsection{Стационарные силы}
\noindent $ F = \v F(\v r, \dot{\v r} $ --- стационарная сила \\
$ W = (\v F, \v v) \leqslant 0 $,~~$\v F(\v r, \dot{\v r})$ --- диссипативная сила \\
\begin{xmp}~
\begin{itemize}
\item $\v F = -kN\frac{\dot{\v r}}{|\dot{\v r}|}$ --- сухое трение
\item $\v F = - \beta \dot{\v r}$ --- вязкое трение
\end{itemize}
\end{xmp}
\noindent $W = (\v F, \v v) \equiv 0$,~~ $\v F$ --- гироскопическая сила \\
\begin{xmp}
$\v F^{\text{кор}} = -m \v w^{\text{кор}} = -2m[\v \omega, \v v]$ \\
$(\v F^{\text{кор}}, \v v) = -2m([\v \omega, \v v], \v v) = 0 $
\end{xmp}

\subsection{Позиционные силы}
$\v F = \v F(r, t)$ --- позиционная сила (силовое поле)

\begin{df} 
$\v F(\v r, t)$ --- потенциальная сила.
\[ \exists u(\v r, t):~~ \v F = \grad_r u \]
$u$ --- силовая функция, $\Pi = -u$ --- потенциальная энергия.
\end{df}

\begin{xmp}
$F = F(x, t)\v e_x = \frac{\partial u}{\partial \v r} = \frac{\partial u}{\partial x} \v e_x + \frac{\partial u}{\partial y} \v e_y$ \\
$ U = \int F(x, t)dx $
\end{xmp}

\begin{df}
Потенциальная сила $\v F(\v r)$ - консервативная.
\end{df}
\begin{xmp} 
$ F = - \frac{\gamma m}{r^2} \cdot \frac{\v r}{r} $ --- консервативная, т.к. \\
$ U = \int(\v F, d\v r) = -\int \frac{\gamma m}{r^3} (\v r, d\v r) = -\int \frac{\gamma m}{r^3} d\lb \frac{(\v r, \v r)}{2} \rb = $ \\
$ = -\int\frac{\gamma m}{r^3} d\frac{r^2}{2} = -\int\frac{\gamma m}{r^2}dr = \frac{\gamma m}{r};~~ n = -\frac{\gamma m}{r}$ \\
$ U = \int (\v F, d\v r)$
\end{xmp}

\subsubsection{Критерий потенциальности}
\begin{ass}
\[ \v F(\v r) = F_x\v e_x + F_y\v e_y + F_z\v e_z \text{ --- потенциальная } \Leftrightarrow 
\begin{cases}
\frac{\partial F_x}{\partial y} = \frac{\partial F_y}{\partial x} \\
\frac{\partial F_y}{\partial z} = \frac{\partial F_z}{\partial y} \\
\frac{\partial F_z}{\partial x} = \frac{\partial F_x}{\partial z} \\
\end{cases} \]
\end{ass}
\begin{proof}~\\

\begin{flalign*}
& \Rightarrow &\\
& u \in c^2 &\\
& \frac{\partial F_x}{\partial y} = \frac{\partial^2 u}{\partial y \partial x} = \frac{\partial^2 u}{\partial x \partial y} = \frac{\partial F_y}{\partial x} &\\
\end{flalign*}

\begin{flalign*}
& \Leftarrow &\\
& u = \int\limits_{\v r_0}^{\v r} F_x(\xi, y, z)d\xi + \int\limits_{\v r_0}^{\v r} F_x(x_0, \eta, z)d\eta + \int\limits_{\v r_0}^{\v r} F_x(x_0, y_0, \zeta)d\zeta &\\
\end{flalign*}
\end{proof}

\begin{cor}
$F(\v r)$ --- потенциальная сила $\Leftrightarrow$ $\oint\limits_C(\v F, d\v r) = 0,~~ \forall C$
\begin{proof}
\[ \oint\limits_{C = \delta W} (\v F, d \v r) = - \int\limits_W(\frac{\partial F_x}{\partial y} - \frac{\partial F_y}{\partial x})dxdy + \ldots = 0 \]
\end{proof}
\end{cor}

Система точек $\v F_i = \v F_i^{(e)} + \v F_i^{(i)}$.

\[ F_i^{(i)} = \sum\limits_{j \neq i} \v F_ij;~ \v F_{ij} = - \v F_{ji} = F_{ij}(|\v r_i - \v r_j|)\frac{\v r_j - \v r_i}{|\v r_j - \v r_i|} \]

\subsection{Свойства внутренних сил}~
\begin{enumerate}
	\item \[ \sum\limits_{i = 1}^N \v F_i^{(i)} = 0 \]
	\begin{proof}
	\[ \sum_{i = 1}^N \v F_i^{(i)} = \sum\limits_{i = 1}^N \sum\limits_{j < i} \v F_{ij} + \sum\limits_{i = 1}^N\sum\limits_{j > i}\v F_{ij} = \sum\limits_{i = 1}^N(\v F_{ij} - \v F_{ji}) = 0 \]
	\end{proof}
	\item \[ \sum\limits_{i = 1}^N[\v r_i, \v F_i^{(i)}] = 0 \]
	\begin{proof}
	\[ \sum\limits_{i = 1}^N \sum\limits_{j < i} [\v r_i, \v F_{ij}] + \sum\limits_{i = 1}^N \sum\limits_{j < i} [\v r_j, \v F_{ij}] = \sum\limits_{i = 1}^N \sum\limits_{j < i} [\v r_i - \v r_j, \v F_{ij}] = 0 \]
	\end{proof}
	\item Внутренние силы потенциальны, т.е.
	\[ \exists u(\v r_1, \ldots, \v r_n):~ \v F_i^{(i)} = grad_{\v r_i} u \]
	\begin{proof}
	\begin{flalign*}
	& u_ij(| \v r |) = \int\limits_0^{|\v r|}F_{ij}(\v \rho)d\rho &\\
	& u = \sum\limits_{i, i < j} u_{ij} ~~~~ \frac{\partial u}{\partial \v r_i} = \sum\limits_{i, i < j} \frac{\partial u_{ij}}{\partial \v r_i} = \sum \frac{\partial u_{ij}}{\partial|\v r_i - \v r_j|} \cdot \frac{\partial|\v r_i - \v r_j|}{\partial \v r_i} &\\
	& |\v r_i - \v r_j| = \sqrt{(x_i - x_j)^2 + (y_i - y_j)^2 + (z_i - z_j)^2} &\\
	& \frac{\partial|\v r_i - \v r_j|}{\partial x_i} = \frac{(x_i - x_j)}{|\v r_i - \v r_j|} ~~~~ \text{Аналогично для $y_i$ и $z_i$} &\\
	& \frac{\partial| \v r_i - \v r_j|}{\partial r_i} = \frac{\v r_i - \v r_j}{|\v r_i - \v r_j|} &\\
	& \frac{\partial u}{\partial \v r_i} = \sum\limits_{i, j,~ i < j}F_{ij}(\v r_i - \v r_j) \cdot \frac{\v r_i - \v r_j}{|\v r_i - \v r_j|} = \v F_i^{(i)} &\\
	\end{flalign*}
	\end{proof}
	\item Работа внутренних сил в тердом теле равна нулю.
	\begin{proof}
	\begin{flalign*}
	& \sum(\v F_i^{(i)}, v_i) = \sum(\v F_i^{(i)}, \v v_s + [\v \omega, \v \rho_i]) = &\\
	& = \left( \underbrace{\sum\v F_i^{(i)}}_{0}, \v v_s \right) + \left(\v \omega, \underbrace{\sum [\v \rho_i, \v F_i^{(i)}]}_0\right) = 0 &\\
	\end{flalign*}
	\end{proof}
\end{enumerate}